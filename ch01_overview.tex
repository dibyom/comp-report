%
% $Id: ch01_overview
%
%   *******************************************************************
%   * SEE THE MAIN FILE "AllegThesis.tex" FOR MORE INFORMATION.       *
%   *******************************************************************

\chapter{Introduction}\label{ch:intro} % we can refer to chapter by the label

%   ************************************************************************
%   * In LaTeX, new paragraphs are begun by simply leaving a blank line in *
%   * the LaTeX file.                                                      *
%   *                                                                      *
%   * The \\ characters should NEVER be used to end a paragraph.           *
%   * They are used only for inserting line breaks in certain situations.  *
%   *                                                                      *
%   * "Widows" (ending paragraph lines at the top of a new page) and       *
%   * "orphans" (opening paragraph lines at the bottom of a page) should   *
%   * be eliminated; this sometimes requires re-writing some of the        *
%   * text to change the line lengths.                                     *
%   ************************************************************************


\section{Importance of Energy Efficiency} \label{sec:motivation}

Energy usage has been continuously increasing since the dawn of the industrial age. In the last 40 years, energy usage globally has risen by approximately 70 percent according to a 2010 World Resources Institute Report. Even more recently, the 2013 International Energy Outlook report projects a 56 percent increase in energy consumption in the next 30 years \cite{ieo2013}. Fossil fuels supply about 80 percent of the world's energy needs with coal and oil together accounting for the vast majority of the number \cite{ipcc}. Rising energy use leads to an increase in greenhouse gas emissions from fossil fuels which further leads to an increase in chances of climate change. The 2007 IPCC report on climate change states that fossil fuels are responsible for 85 percent of human generated CO\textsubscript{2} emissions. This figure is stated to increase by another 62 percent by 2030 from 2002 levels.

Energy efficiency, i.e., ``delivering the same (or more) services for less energy'' \cite{epa}, is one of the quickest and cheapest ways to increase the amount of energy available for use. The European Council for an Energy Efficient Economy describes energy efficiency  as the `cornerstone of a sustainable society' \cite{ecees}.A large share of the energy supply has to come from renewable energy sources such as wind and solar power in order to comply with international treaties such as the Kyoto Protocol. However, with increasing energy demand, the development of renewables needs to be supplemented with energy efficiency.

From an environmental perspective, an obvious benefit to energy efficiency is 
the reduction in the amount of greenhouse gas emissions. Reduction in greenhouse gas emissions can help with  improving urban air quality, reducing acid rain, and reducing eutrophication (i.e. an increase in the concentration of nutrients in water that promotes excessive algae growth) \cite{ecees}. There are economic benefits to energy efficiency as well. According to the Department of Energy's Energy Efficiency and Renewable Energy program, Americans saved \$7 billion on residential energy bills in 2004 from energy saving measures and by building energy efficient homes \cite{wri}. Increased energy efficiency contributes to energy security by making a country less dependent on imported non-renewable energy sources and thereby making it more competitive in the globalized world.

The environmental impact of buildings are significant. In developed countries, buildings accoun for 70 percent of electricity consumption and 30 percent of greenhouse gas emission during their operational phase \cite{Castro-Lacouture2009}. In addition, other phases such as construction of the building, transportation and extraction of raw materials use significant amounts of energy as well \cite{Castro-Lacouture2009}. In the United States, buildings consume about 70 percent of the total electricity and about 39 percent of all energy consumption \cite{Wang2005a}. Energy efficiency in buildings can thus play an important role in making the building `green'. Some such measures are structural and can only be included in newly constructed buildings; many others can be incorporated during building refurbishment. While there are software tools available to simulate the effects and impacts of a particular design, tools for optimizing the design are not readily available \cite{Wang2005b} \cite{Pernodet2009}.

\begin{figure}[htbp]
\centering
\includegraphics[width=0.8\linewidth,scale=0.6]{images/energy.png}
\caption{Buildings Share of U.S. Primary Energy Consumption 
Uses \cite{wri}}
\label{fig:energy}
\end{figure}

\section{Optimization techniques}\label{sec:stateofart}
Solving an optimization problem allows us to find at least one solution that 
minimizes or maximizes a particular criterion. This criterion is represented by an objective or a fitness function that depends on variable parameters (both continuous and discrete) that describe the solutions \cite{Pernodet2009}. 
Optimizations can be both single criterion or multi criteria. There are three 
main methods for optimization: enumerative, calculus based, and random \cite{Pernodet2009}. Enumerative methods go through every single solution in the search space in order to find the optimal one. While this method is simple, it is extremely inefficient especially when it comes to problems such as building optimization due to the huge number of possible solutions. Calculus based methods use a rigorous mathematical expression of the objective function. The main limitation to this method, besides having to know an explicit mathematical expression (that sometimes has to be continuous), is that it can find a local optimum in the neighborhood without going through the global search space. Random methods, as the name suggests, use random evaluation of solutions and are often built to emulate other phenomena.

\begin{figure}[htbp]
\centering
\includegraphics[width =0.7\linewidth]{images/pernodet.png}
\caption{Representing solutions as chromosome \cite{Pernodet2009}}
\label{fig:pernodet}
\end{figure}


Genetic algorithms (GAs) are a form of random optimization methods that seek to mimic the process of evolution in order to select an optimal solution. Strings of either real numbers or  bits (binary digits, 0 and 1) are used analogous to a gene in order to represent the parameters (Fig.\ref{fig:pernodet}) \cite{Pernodet2009} \cite{Coley2002}. Multiple such sub-strings are then concatenated to form the genotype. An initial population of these genotypes is generated randomly. The algorithm consists of three main functions: selection, mutation and crossover (Fig.\ref{fig:pernodet2}). During selection, the fitness of an individual string is evaluated using the parameter values it represents. The fittest of the strings are then crossed-over, i.e., allowed to mate and produce progeny by combining substrings of random length from pairs of genotypes. Mutation is allowed by occasional changing the values of a string position of a newly created progeny. After a number of generations of the process, the parameter values represented by the genotypes hopefully converge towards optimal solution values \cite{Coley2002}.

\begin{figure}[htbp]
\centering
\includegraphics[width =0.7\linewidth]{images/pernodet2.png}
\caption{Crossover and Mutation operators \cite{Pernodet2009}}
\label{fig:pernodet2}
\end{figure}


The general advantages of using genetic algorithms for optimization problems are all relevant in the case of energy optimization for buildings. GAs do not require a knowledge of the mathematical structure of the problem. A GA search is not limited to a local optimum \cite{Pernodet2009}. Further, multi-objective GAs such as the non-dominated sorting genetic algorithm (NSGA-II) produce Pareto optimal solutions. A solution is Pareto optimal (or non-dominated) if a decrease in one objective cannot happen without an increase in at least one other objective \cite{Deb2002}\cite{Pernodet2009}. For a problem with two objectives (such as energy consumption and cost), the result is a curve of Pareto solutions instead of just one solution. This allows us to obtain a whole set of solutions from which we can then choose (Fig.\ref{fig:pareto}). %Add more about MOGA%

\begin{figure}[htbp]
\centering
\includegraphics[width = 0.5\linewidth]{images/pareto.png}
\caption{Pareto Optimal Solution Curve \cite{Coley2002}}
\label{fig:pareto}
\end{figure}

\section{Goals of the Project}\label{sec:goals}

The aim of this proposed project is to use a genetic algorithm-based approach to create a system for optimizing energy efficiency in a building. The system will take in parameters needed to model the building design and output a set of Pareto optimal solutions. This will be a multi-objective optimization problem with two objectives\textemdash minimizing energy usage and minimizing associated cost. A multi-objective genetic algorithm, NSGA-II, will be used along with a building energy simulation program, EnergyPlus, for fitness evaluation. The next chapter provides a more detailed overview of genetic algorithms in general and NSGA-II in particular. The following chapter provides a comparative discussion of five studies specific to building optimization using genetic algorithms. Next, I will discuss the implementation details of this project and present a case study utilizing it. Finally, the conclusion will provide a summary of the goals achieved, describe threats to the validity of this project and list ways in which future work can extend the project. 

